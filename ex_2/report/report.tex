\documentclass[12pt, a4paper, oneside]{ctexart}
\usepackage{float, amsmath, amsthm, amssymb, appendix, bm, graphicx, hyperref, mathrsfs, listings, xcolor, makecell, adjustbox}

\title{\textbf{论文标题}}  
\author{黄嘉浩 \quad 电子工程系 \quad 2022010666}
\date{\today}
\linespread{1.5}
\newtheorem{theorem}{定理}[section]
\newtheorem{definition}[theorem]{定义}
\newtheorem{lemma}[theorem]{引理}
\newtheorem{corollary}[theorem]{推论}
\newtheorem{example}[theorem]{例}
\newtheorem{proposition}[theorem]{命题}
\renewcommand{\abstractname}{\Large\textbf{摘要}}

\lstset{
    basicstyle=\scriptsize, % Change font size to footnotesize
    breaklines,%自动换行
    columns=flexible,%不随便添加空格,只在已经有空格的地方添加空格,
    numbers=left, % 显示行号
    numberstyle=\tiny, % 行号字体
    keywordstyle=\color{blue!70}, % 关键字颜色
    commentstyle=\color{red!50!green!50!blue!50}, % 注释颜色
    frame=shadowbox, % 为代码块添加阴影框
    rulesepcolor=\color{red!20!green!20!blue!20}, % 阴影框颜色
    escapeinside=``, % 允许在代码块中使用 LaTeX 命令
    xleftmargin=0em, xrightmargin=0em, aboveskip=1em, % 设置代码块的边距
    framexleftmargin=2em % 阴影框左边距
} 

\begin{document}

\begin{titlepage}
    \vspace*{\fill}  % 顶部留白
    \begin{center}
        \fontsize{32}{18}\selectfont\textbf{银行家算法} \\
        \vspace{0.5cm}
        \fontsize{20}{18}\selectfont{\textbf{——《操作系统》课程实验四:}} \\
        \vspace{0.5cm}
        \fontsize{20}{18}\selectfont{\textbf{处理机调度问题}} \\

        \vspace{2cm}
        \fontsize{18}{18}\selectfont\textbf{黄嘉浩 \quad 无27 \quad 2022010666}  \\
        \vspace{0.5cm}
        \fontsize{18}{18}\selectfont{huang-jh22@mails.tsinghua.edu.cn}  \\
        \vspace{0.5cm}
        \fontsize{18}{18}\selectfont{\today}\\
    \end{center}
    \vspace*{\fill}  % 底部留白
\end{titlepage}

% 目录
\newpage
\pagenumbering{Roman}
\setcounter{page}{1}
\tableofcontents

% 正文
\newpage
\setcounter{page}{1}
\pagenumbering{arabic}

\section{问题描述及实现要求}

\textbf{问题描述}

银行家算法是避免死锁的一种重要方法,将操作系统视为银行家,操作系统管理的资源
视为银行家管理的资金,进程向操作系统请求分配资源即企业向银行申请贷款的过程。

请根据银行家算法的思想,编写程序模拟实现动态资源分配,并能够有效避免死锁的发
生。

\textbf{实现要求}

1. 对实现的算法通过流程图进行说明;

2. 设计不少于三组测试样例,需包括资源分配成功和失败的情况;

3. 能够展示系统资源占用和分配情况的变化及安全性检测的过程;

4. 结合操作系统课程中银行家算法理论对实验结果进行分析,验证结果的正确性;

5. 分析算法的鲁棒性及算法效率。

\section{实验环境}

\textbf{编程语言} \quad Python 3.12

\textbf{编程环境} \quad Windows 11

\section{设计思路及程序结构}

\subsection{设计思路与核心流程}

为了保证所有进程正常运行完成,在系统运行过程中,对进程提出的每一个(系统能够满足的)资源申请
进行动态检查(安全性算法),并根据检查结果决定是否分配资源,若
分配后系统可能发生死锁,则不予分配,否则予以分配。

安全性算法的步骤是,遍历每个进程的需求资源量,判断当前系统的动态可分配资源是否满足该进程的需求量。
若满足,则将该进程的资源分配给它,并将其状态改为完成状态(True),释放它所占有的资源;
否则,继续遍历下一个进程。
若所有进程都处于完成状态,则说明系统处于安全状态。否则,系统处于不安全状态。

\textbf{核心流程}

\begin{figure}[H]
    \centering
    \includegraphics[width=0.6\textwidth]{flowchart.png}
    \caption{银行家算法核心流程}
    \label{fig:flowchart}
\end{figure}

\subsection{程序结构与关键代码解释}

银行家算法的数据结构共有四类,如下所示。

\begin{table}[H]
    \centering
    \renewcommand{\arraystretch}{1.5} % Adjust row height for better readability
    \setlength{\tabcolsep}{4pt} % Adjust column spacing

    \adjustbox{max height=\textheight, max width=\textwidth}{%
    \begin{tabular}{|c|l|}
    \hline
    \makecell{\textbf{类别}}      & \makecell{\textbf{功能}} \\ \hline
    \makecell{可利用资源向量\\available} & \makecell{含有m个元素的数组,\\其中每个元素代表一类可利用资源的数目}     \\ \hline
    \makecell{最大需求矩阵\\max}        & \makecell{n*m矩阵,\\表示n个进程的每一个对m类资源的最大需求}       \\ \hline
    \makecell{分配矩阵\\allocation}   & \makecell{n*m矩阵,\\表示每个进程已分配的每类资源的数目}          \\ \hline
    \makecell{需求矩阵\\need}         & \makecell{n*m矩阵,\\表示每个进程还需要各类资源数}             \\ \hline
    \end{tabular}
    }
\end{table}

\subsubsection{安全性算法}

\begin{lstlisting}[language=Python, caption=安全性算法代码]
def is_safe_state(self):
    work = self.available.copy() #动态可分配资源
    finish = [False] * self.num_processes # 是否有足够的资源分配给进程,初值为 false
    safe_sequence = []
    
    while True:
        found = False
        for i in range(self.num_processes):
            if not finish[i] and np.all(self.need[i] <= work):
                work += self.allocation[i]
                finish[i] = True
                safe_sequence.append(i)
                found = True
                break
        if not found:
            break
\end{lstlisting}

该安全性算法的主要流程如下:

\begin{enumerate}
    \item 首先,复制当前可用资源向量 \texttt{available} 到临时变量 \texttt{work},并初始化所有进程的完成标志 \texttt{finish} 为 \texttt{False},表示所有进程都尚未完成。
    \item 进入循环,每次循环尝试寻找一个尚未完成且其需求 \texttt{need[i]} 小于等于当前可用资源 \texttt{work} 的进程 $i$。
    \item 如果找到这样的进程 $i$,则认为该进程可以顺利完成,将其分配的资源 \texttt{allocation[i]} 归还给系统(即加到 \texttt{work} 上),并将 \texttt{finish[i]} 置为 \texttt{True},同时将该进程编号加入安全序列 \texttt{safe\_sequence}。
    \item 如果本轮循环没有找到任何可以完成的进程(即 \texttt{found} 仍为 \texttt{False}),则跳出循环。
    \item 最后,判断所有进程的 \texttt{finish} 是否都为 \texttt{True}。若是,则系统处于安全状态,并返回安全序列;否则,系统处于不安全状态。
\end{enumerate}

\subsubsection{资源申请算法}

\begin{lstlisting}[language=Python, caption=资源申请算法代码]
def request_resources(self, process_id, request): 
    request = np.array(request)
    
    # 检查请求是否超过需求
    if np.any(request > self.need[process_id]):
        print(f"错误:进程 {process_id} 的请求超过了其最大需求")
        return False
        
    # 检查请求是否超过可用资源
    if np.any(request > self.available):
        print(f"错误:进程 {process_id} 的请求超过了可用资源")
        return False
        
    # 尝试分配资源
    old_available = self.available.copy()
    old_allocation = self.allocation.copy()
    old_need = self.need.copy()
    
    self.available -= request
    self.allocation[process_id] += request
    self.need[process_id] -= request
    
    # 检查安全性
    safe, sequence = self.is_safe_state()  
\end{lstlisting}

该资源申请算法的主要流程如下:

\begin{enumerate}
    \item 首先,将请求向量 \texttt{request} 转换为数组,方便后续计算。
    \item 检查请求是否超过了该进程的最大需求(\texttt{need[process\_id] })。如果超过,直接拒绝请求并返回 \texttt{False}。
    \item 检查请求是否超过了当前系统可用资源(\texttt{available})。如果超过,直接拒绝请求并返回 \texttt{False}。
    \item 如果请求合法,尝试分配资源。为防止分配后系统进入不安全状态,先保存当前的 \texttt{available}、\texttt{allocation} 和 \texttt{need} 状态,便于回滚。
    \item 执行资源分配操作:将请求的资源从 \texttt{available} 中减去,分配到对应进程的 \texttt{allocation},并更新 \texttt{need}。
    \item 调用前述安全性算法(\texttt{is\_safe\_state})判断分配后系统是否安全。
    \item 如果系统安全,则分配成功,返回 \texttt{True};否则,回滚到分配前的状态,拒绝本次请求,返回 \texttt{False}。
\end{enumerate}

这样,便保证了每次资源分配都不会让系统进入不安全状态,从而有效避免死锁的发生。

\section{程序运行情况}

笔者设计了三组测试用例,分别展示了资源分配成功、资源分配失败(超过需求及超过可用资源导致不安全状态)和多资源类型的情况。

\subsection{测试用例一:资源分配成功}

\subsubsection{测试用例一分析}

\textbf{基本信息:}

\begin{itemize}
    \item 进程数:5
    \item 资源种类数:3
    \item 总资源向量:\texttt{[10, 5, 7]}
\end{itemize}

\textbf{分配矩阵(allocation):}

\begin{table}[H]
    \centering
    \begin{tabular}{c|ccc}
        \hline
        进程 & A & B & C \\
        \hline
        P0 & 0 & 1 & 0 \\
        P1 & 2 & 0 & 0 \\
        P2 & 3 & 0 & 2 \\
        P3 & 2 & 1 & 1 \\
        P4 & 0 & 0 & 2 \\
        \hline
    \end{tabular}
\end{table}

\textbf{最大需求矩阵(max):}

\begin{table}[H]
    \centering
    \begin{tabular}{c|ccc}
        \hline
        进程 & A & B & C \\
        \hline
        P0 & 7 & 5 & 3 \\
        P1 & 3 & 2 & 2 \\
        P2 & 9 & 0 & 2 \\
        P3 & 2 & 2 & 2 \\
        P4 & 4 & 3 & 3 \\
        \hline
    \end{tabular}
\end{table}

\textbf{请求序列:}

\begin{enumerate}
    \item 进程1请求\texttt{[1, 0, 2]}
    \item 进程4请求\texttt{[3, 3, 0]}
    \item 进程0请求\texttt{[0, 2, 0]}
\end{enumerate}

% 你可以用如下格式展示每一步的分析过程
\textbf{算法执行过程分析:}

\begin{itemize}
    \item 初始可用资源向量:\texttt{[3, 3, 2]}
    \item 进程1请求\texttt{[1, 0, 2]},满足需求且系统安全,分配成功,此时找到一条系统安全序列:\texttt{[1, 3, 0, 2, 4]}。
    \item 分配后可用资源向量:\texttt{[2, 3, 0]}
    \item 进程4请求\texttt{[3, 3, 0]},请求超过可用资源,分配失败。
    \item 进程0请求\texttt{[0, 2, 0]},资源分配会导致不安全状态(死锁),分配失败。
    \item 最终系统安全序列:\texttt{[1, 3, 0, 2, 4]}。
\end{itemize}

\subsubsection{测试用例一运行结果}

\begin{lstlisting}[language=Python, caption=测试用例一运行结果]
===== 测试用例 资源分配成功案例 =====

-> 当前系统状态:
可用资源: [3 3 2]

分配矩阵:
进程    资源0   资源1   资源2
0       0       1       0
1       2       0       0
2       3       0       2
3       2       1       1
4       0       0       2

需求矩阵:
进程    资源0   资源1   资源2
0       7       4       3
1       1       2       2
2       6       0       0
3       0       1       1
4       4       3       1

系统处于安全状态,安全序列为: [1, 3, 0, 2, 4]

进程 1 请求资源: [1, 0, 2]
资源分配成功,安全序列为: [1, 3, 0, 2, 4]

-> 当前系统状态:
可用资源: [2 3 0]

分配矩阵:
进程    资源0   资源1   资源2
0       0       1       0
1       3       0       2
2       3       0       2
3       2       1       1
4       0       0       2

需求矩阵:
进程    资源0   资源1   资源2
0       7       4       3
1       0       2       0
2       6       0       0
3       0       1       1
4       4       3       1

系统处于安全状态,安全序列为: [1, 3, 0, 2, 4]

进程 4 请求资源: [3, 3, 0]
错误:进程 4 的请求超过了可用资源

-> 当前系统状态:
可用资源: [2 3 0]

分配矩阵:
进程    资源0   资源1   资源2
0       0       1       0
1       3       0       2
2       3       0       2
3       2       1       1
4       0       0       2

需求矩阵:
进程    资源0   资源1   资源2
0       7       4       3
1       0       2       0
2       6       0       0
3       0       1       1
4       4       3       1

系统处于安全状态,安全序列为: [1, 3, 0, 2, 4]

进程 0 请求资源: [0, 2, 0]
资源分配会导致不安全状态,请求被拒绝

-> 当前系统状态:
可用资源: [2 3 0]

分配矩阵:
进程    资源0   资源1   资源2
0       0       1       0
1       3       0       2
2       3       0       2
3       2       1       1
4       0       0       2

需求矩阵:
进程    资源0   资源1   资源2
0       7       4       3
1       0       2       0
2       6       0       0
3       0       1       1
4       4       3       1

系统处于安全状态,安全序列为: [1, 3, 0, 2, 4]
\end{lstlisting}

最终,系统找到了安全序列 \texttt{[1, 3, 0, 2, 4]},说明系统处于安全状态,这是符合前述分析的。

\subsection{测试用例二:资源分配失败}

\subsubsection{测试用例二分析}

\textbf{基本信息:}
\begin{itemize}
    \item 进程数:5
    \item 资源种类数:3
    \item 总资源向量:\texttt{[10, 5, 7]}
\end{itemize}

\textbf{分配矩阵(allocation):}
\begin{table}[H]
    \centering
    \begin{tabular}{c|ccc}
        \hline
        进程 & A & B & C \\
        \hline
        P0 & 0 & 4 & 0 \\
        P1 & 2 & 0 & 0 \\
        P2 & 3 & 0 & 2 \\
        P3 & 2 & 1 & 1 \\
        P4 & 0 & 0 & 2 \\
        \hline
    \end{tabular}
\end{table}

\textbf{最大需求矩阵(max):}
\begin{table}[H]
    \centering
    \begin{tabular}{c|ccc}
        \hline
        进程 & A & B & C \\
        \hline
        P0 & 7 & 5 & 3 \\
        P1 & 3 & 2 & 2 \\
        P2 & 9 & 0 & 2 \\
        P3 & 2 & 2 & 2 \\
        P4 & 4 & 3 & 3 \\
        \hline
    \end{tabular}
\end{table}

\textbf{请求序列:}
\begin{enumerate}
    \item 进程1请求\texttt{[3, 0, 2]}
    \item 进程4请求\texttt{[3, 0, 3]}
    \item 进程0请求\texttt{[0, 0, 2]}
\end{enumerate}

\textbf{算法执行过程分析:}
\begin{itemize}
    \item 初始可用资源向量:\texttt{[3, 0, 2]}
    \item 进程1请求\texttt{[3, 0, 2]},请求量超过其最大需求\texttt{[1, 2, 2]},系统拒绝分配。
    \item 进程4请求\texttt{[3, 0, 3]},请求量超过当前可用资源,系统拒绝分配。
    \item 进程0请求\texttt{[0, 0, 2]},虽然请求量合法且不超过可用资源,但分配后系统进入不安全状态,系统拒绝分配。
    \item 最终系统状态未发生变化,系统始终处于不安全状态。
\end{itemize}

\subsubsection{测试用例二运行结果}
\begin{lstlisting}[language=Python, caption=测试用例二运行结果]
===== 测试用例 资源分配失败案例 =====

-> 当前系统状态:
可用资源: [3 0 2]

分配矩阵:
进程    资源0   资源1   资源2
0       0       4       0
1       2       0       0
2       3       0       2
3       2       1       1
4       0       0       2

需求矩阵:
进程    资源0   资源1   资源2
0       7       1       3
1       1       2       2
2       6       0       0
3       0       1       1
4       4       3       1

系统处于不安全状态

进程 1 请求资源: [3, 0, 2]
错误:进程 1 的请求超过了其最大需求

-> 当前系统状态:
可用资源: [3 0 2]

分配矩阵:
进程    资源0   资源1   资源2
0       0       4       0
1       2       0       0
2       3       0       2
3       2       1       1
4       0       0       2

需求矩阵:
进程    资源0   资源1   资源2
0       7       1       3
1       1       2       2
2       6       0       0
3       0       1       1
4       4       3       1

系统处于不安全状态

进程 4 请求资源: [3, 0, 3]
错误:进程 4 的请求超过了其最大需求

-> 当前系统状态:
可用资源: [3 0 2]

分配矩阵:
进程    资源0   资源1   资源2
0       0       4       0
1       2       0       0
2       3       0       2
3       2       1       1
4       0       0       2

需求矩阵:
进程    资源0   资源1   资源2
0       7       1       3
1       1       2       2
2       6       0       0
3       0       1       1
4       4       3       1

系统处于不安全状态

进程 0 请求资源: [0, 0, 2]
资源分配会导致不安全状态,请求被拒绝

-> 当前系统状态:
可用资源: [3 0 2]

分配矩阵:
进程    资源0   资源1   资源2
0       0       4       0
1       2       0       0
2       3       0       2
3       2       1       1
4       0       0       2

需求矩阵:
进程    资源0   资源1   资源2
0       7       1       3
1       1       2       2
2       6       0       0
3       0       1       1
4       4       3       1

系统处于不安全状态
\end{lstlisting}

最终,系统无法找到安全序列,这是符合前述分析的。

\subsection{测试用例三:多资源类型}

\subsubsection{测试用例三分析}

\textbf{基本信息:}
\begin{itemize}
    \item 进程数:4
    \item 资源种类数:4
    \item 总资源向量:\texttt{[10, 5, 7, 8]}
\end{itemize}

\textbf{分配矩阵(allocation):}
\begin{table}[H]
    \centering
    \begin{tabular}{c|cccc}
        \hline
        进程 & A & B & C & D \\
        \hline
        P0 & 0 & 1 & 1 & 0 \\
        P1 & 2 & 0 & 0 & 1 \\
        P2 & 3 & 0 & 2 & 0 \\
        P3 & 2 & 1 & 1 & 2 \\
        \hline
    \end{tabular}
\end{table}

\textbf{最大需求矩阵(max):}
\begin{table}[H]
    \centering
    \begin{tabular}{c|cccc}
        \hline
        进程 & A & B & C & D \\
        \hline
        P0 & 7 & 5 & 3 & 2 \\
        P1 & 3 & 2 & 2 & 2 \\
        P2 & 9 & 0 & 2 & 2 \\
        P3 & 2 & 2 & 2 & 4 \\
        \hline
    \end{tabular}
\end{table}

\textbf{请求序列:}
\begin{enumerate}
    \item 进程1请求\texttt{[1, 0, 1, 0]}
    \item 进程3请求\texttt{[0, 1, 0, 1]}
    \item 进程2请求\texttt{[2, 0, 0, 1]}
\end{enumerate}

\textbf{分析过程:}
\begin{itemize}
    \item 初始可用资源向量:\texttt{[3, 3, 3, 5]}
    \item 进程1请求\texttt{[1, 0, 1, 0]},当前可用资源满足其需求,资源分配后系统存在安全序列\texttt{[1, 3, 0, 2]},资源分配成功。
    \item 分配后可用资源向量:\texttt{[2, 3, 2, 5]}
    \item 进程3请求\texttt{[0, 1, 0, 1]},当前可用资源满足其需求,资源分配后系统存在安全序列\texttt{[1, 3, 0, 2]},资源分配成功。
    \item 分配后可用资源向量:\texttt{[2, 2, 2, 4]}
    \item 进程2请求\texttt{[2, 0, 0, 1]},当前可用资源满足其需求,资源分配后系统存在安全序列\texttt{[1, 3, 0, 2]},资源分配成功。
    \item 分配后可用资源向量:\texttt{[0, 2, 2, 3]}
    \item 最终系统安全序列:\texttt{[1, 3, 0, 2]}。
\end{itemize}

\subsubsection{测试用例三运行结果}

\begin{lstlisting}[language=Python, caption=测试用例三运行结果]
===== 测试用例 多资源类型案例 =====

-> 当前系统状态:
可用资源: [3 3 3 5]

分配矩阵:
进程    资源0   资源1   资源2   资源3
0       0       1       1       0
1       2       0       0       1
2       3       0       2       0
3       2       1       1       2

需求矩阵:
进程    资源0   资源1   资源2   资源3
0       7       4       2       2
1       1       2       2       1
2       6       0       0       2
3       0       1       1       2

系统处于安全状态,安全序列为: [1, 3, 0, 2]

进程 1 请求资源: [1, 0, 1, 0]
资源分配成功,安全序列为: [1, 3, 0, 2]

-> 当前系统状态:
可用资源: [2 3 2 5]

分配矩阵:
进程    资源0   资源1   资源2   资源3
0       0       1       1       0
1       3       0       1       1
2       3       0       2       0
3       2       1       1       2

需求矩阵:
进程    资源0   资源1   资源2   资源3
0       7       4       2       2
1       0       2       1       1
2       6       0       0       2
3       0       1       1       2

系统处于安全状态,安全序列为: [1, 3, 0, 2]

进程 3 请求资源: [0, 1, 0, 1]
资源分配成功,安全序列为: [1, 3, 0, 2]

-> 当前系统状态:
可用资源: [2 2 2 4]

分配矩阵:
进程    资源0   资源1   资源2   资源3
0       0       1       1       0
1       3       0       1       1
2       3       0       2       0
3       2       2       1       3

需求矩阵:
进程    资源0   资源1   资源2   资源3
0       7       4       2       2
1       0       2       1       1
2       6       0       0       2
3       0       0       1       1

系统处于安全状态,安全序列为: [1, 3, 0, 2]

进程 2 请求资源: [2, 0, 0, 1]
资源分配成功,安全序列为: [1, 3, 2, 0]

-> 当前系统状态:
可用资源: [0 2 2 3]

分配矩阵:
进程    资源0   资源1   资源2   资源3
0       0       1       1       0
1       3       0       1       1
2       5       0       2       1
3       2       2       1       3

需求矩阵:
进程    资源0   资源1   资源2   资源3
0       7       4       2       2
1       0       2       1       1
2       4       0       0       1
3       0       0       1       1

系统处于安全状态,安全序列为: [1, 3, 2, 0]
\end{lstlisting}

最终,系统找到了安全序列 \texttt{[1, 3, 2, 0]},说明系统处于安全状态,这是符合前述分析的。

\section{算法鲁棒性及效率分析}

本算法在资源请求时,首先检查请求是否超过进程的最大需求(need)和系统当前可用资源(available)。若请求不合法,立即拒绝,防止非法操作导致系统异常。这保证了算法对异常输入的容错能力。

在尝试分配资源后,算法会调用安全性检测(is\_safe\_state)。若检测到系统进入不安全状态,会将所有资源分配操作回滚到分配前的状态,确保系统始终处于安全状态。这一机制极大增强了算法对潜在死锁和资源分配错误的防护能力。

对于超过最大需求或可用资源的请求,算法会输出明确的错误提示,便于用户定位问题。

总之,在输入合法性检查、资源分配回滚和错误提示等方面,算法展现了较强的鲁棒性。

\textbf{时间复杂度分析}

在每次资源请求时,算法需要进行两步主要操作:
(1)合法性检查(与进程数和资源种类线性相关);
(2)安全性检测(is\_safe\_state),其核心是尝试找到一个安全序列。
安全性检测的最坏情况是每次都只能完成一个进程,总共需要遍历 $n$ 次($n$ 为进程数),每次判断需遍历所有进程和资源,因此时间复杂度为 $O(n^2 \cdot m)$,其中 $m$ 为资源种类数。
整体来看,单次资源请求的复杂度为 $O(n^2 \cdot m)$,对于中小规模系统(进程数和资源种类不大)是可以接受的。

\textbf{空间复杂度分析}

算法主要维护分配矩阵、最大需求矩阵、需求矩阵和可用资源向量,空间复杂度为 $O(n \cdot m)$。

\section{思考题}

\textbf{Q1} \quad 银行家算法在实现过程中需注意资源分配的哪些事项才能避免死锁?

\begin{enumerate} 
    \item \textbf{请求合法性检查}:每次进程请求资源时,必须首先判断请求量是否超过其最大需求(need)以及当前系统可用资源(available)。只有在请求合法且系统有足够资源时,才考虑分配。 
    \item \textbf{安全性检测}:在资源实际分配前,需通过安全性算法(is\_safe\_state)模拟分配,判断分配后系统是否仍处于安全状态。只有在分配后系统依然安全时,才真正分配资源。 
    \item \textbf{状态回滚机制}:若分配后系统不安全,必须将所有资源分配操作回滚到分配前的状态,确保系统不会进入不安全状态。 
    \item \textbf{动态更新数据结构}:每次资源分配或释放后,需及时更新分配矩阵(allocation)、需求矩阵(need)和可用资源向量(available),保证数据的准确性。 
\end{enumerate}

只有严格遵循上述事项,才能确保银行家算法在动态资源分配过程中有效避免死锁,保证系统安全运行。

\section{实验总结}

本实验实现了银行家算法,模拟了资源分配和安全性检测的过程。通过多组测试用例验证了算法的正确性和鲁棒性。
实验中,笔者深入理解了银行家算法的核心思想和实现细节,并掌握了如何在实际系统中应用该算法来避免死锁和确保系统安全。

总体来看,实验过程进展顺利。本实验采用的银行家算法步骤参考了操作系统课件,遇到的问题主要集中在算法实现的细节上,如资源请求的合法性检查和安全性检测的实现。
由于 Python numpy 库直接支持矩阵向量乘法(MVM)运算,故笔者采用 Python 语言编写了本实验。
通过不断调试和测试,最终成功实现了预期功能。

最后,感谢老师与助教的辛勤付出,使得笔者获益匪浅。再次对老师和助
教表示衷心感谢!
\end{document}